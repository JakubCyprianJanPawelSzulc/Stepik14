\documentclass{beamer}

\usepackage[polish]{babel}
\usepackage[utf8]{inputenc}
\usepackage[T1]{fontenc}
\usepackage{wrapfig}
\usetheme{Warsaw}
\usecolortheme{beetle}

\title{Różne graty}
\author{Jakub Szulc}

\begin{document}

\begin{frame}
\titlepage
\end{frame}

\begin{frame}{Fabryka Samochodów Osobowych}
Polskie przedsiębiorstwo przemysłu motoryzacyjnego produkujące samochody osobowe zlokalizowane w Warszawie w dzielnicy Praga-Północ na Pelcowiźnie zbudowane od podstaw na przełomie lat 40. i 50. XX wieku. W latach 1986–2003 w skład FSO wchodził Zakład Samochodów Dostawczych w Nysie.\cite{Wikipedia}
\pause
\begin{center}
\includegraphics[scale=0.15]{1.jpg}\cite{zdj1}
\end{center}
\end{frame}

\begin{frame}{Najlepszy produkt FSO}

\begin{wrapfigure}{l}{0.37\textwidth}
\includegraphics[scale=0.11]{2.jpg}\\
Atu Plus najlepsza buda \cite{zdj2}\footnote[1]{To zdjęcie przedstawia Bolognesa Atu Pluise}
\end{wrapfigure}
FSO Polonez – samochód osobowy produkowany przez Fabrykę Samochodów Osobowych w Warszawie od 3 maja 1978 roku do 22 kwietnia 2002 roku. Powstał jako następca Polskiego Fiata 125p, który był jednak produkowany równolegle aż do 1991 roku. Samochód przeszedł kilka większych modernizacji, wprowadzono także kolejne odmiany.\cite{Wikipedia}
\end{frame}

\begin{frame}{Plusy i minusy Polonezów Plusów}
\begin{table}[]
\begin{tabular}{|l|l|}
\hline
\multicolumn{1}{|c|}{\textbf{+}}                                                                          & \multicolumn{1}{c|}{\textbf{-}}                                                                                                                   \\ \hline
\begin{tabular}[c]{@{}l@{}}Ten w nazwie\\ \end{tabular} & \begin{tabular}[c]{@{}l@{}}Nietrwałość\end{tabular} \\ \hline
\begin{tabular}[c]{@{}l@{}}Ten na akumulatorze\end{tabular}                             & Bylejakość wykonania \\ \hline
\end{tabular}
\footnote[1]{Proste fakty}
\end{table}
\end{frame}

\begin{frame}{Taktyczny Polonaise Truck}
\begin{center}
\includegraphics[scale=0.2]{3.jpg}\cite{zdj3}\footnote[1]{Nie wiem nawet jak to skomentować}
\end{center}
\end{frame}

\begin{frame}{Inne produkty FSO/Daewoo-FSO}
\begin{enumerate}
\item Syrena
\pause
\item 125p
\pause
\item Daewoo Lanos
\pause
\item Daewoo Matiz
\pause
\item Chevrolet Aveo
\end{enumerate}
\end{frame}

\begin{frame}{Syrena}
To nawet nie jest samochód
\pause
\begin{center}
\includegraphics[scale=0.25]{4.jpg}\cite{zdj4}\footnote[1]{Śmiechu warte}
\end{center}
\end{frame}

\begin{frame}{125p}
Polski Fiat 125p – samochód osobowy klasy średniej, produkowany w Polsce w FSO w Warszawie od 1967 do 1991 roku na podstawie licencji włoskiej firmy FIAT. Po wygaśnięciu licencji w 1983 roku nazwę zmieniono na FSO 125p. Spotykane są też oznaczenia FSO 1300/1500. Potocznie nazywany Dużym Fiatem.\cite{Wikipedia}
\pause
\begin{center}
\includegraphics[scale=0.5]{5.jpg}\cite{zdj5}\footnote[1]{Jamnior}
\end{center}
\end{frame}

\begin{frame}{Daewoo Lanos}
 Daewoo Lanos został zaprezentowany w Polsce po raz pierwszy w 1997 roku.
Montaż Lanosa w Polsce z uproszczonych zestawów SKD w Daewoo-FSO rozpoczęto 30 września 1997 roku. Początkowo stosowano wyłącznie silniki 1,5 8V oraz 1,6 16V, a udział krajowych komponentów wynosił zaledwie 6 procent\cite{Wikipedia}
\pause
\begin{center}
\includegraphics[scale=0.13]{6.jpg}\cite{zdj6}
\end{center}
\end{frame}

\begin{frame}{Daewoo Matiz}
Daewoo Matiz - samochód osobowy klasy najmniejszej produkowany pod południowokoreańską marką Daewoo w latach 1998-2008, 1998-2011 w Korei Południowej oraz pod polską marką FSO jako FSO Matiz w latach 2004-2007.\cite{Wikipedia}
\pause
\begin{center}
\includegraphics[scale=0.2]{7.jpg}\cite{zdj7}\footnote[1]{To jest ten wzmocniony}
\end{center}
\end{frame}

\begin{frame}{Chevrolet Aveo}
W październiku 2007 roku rozpoczęła się produkcja Chevroleta Aveo z linii T250 w zakładach FSO w Warszawie. Pierwsze pojazdy próbne zjechały z taśm 11 lipca 2007 roku, natomiast oficjalna produkcja rozpoczęła się 6 listopada 2007 roku. Do kwietnia 2008 roku Aveo produkowane przez FSO dostarczane było wyłącznie na rynek ukraiński. Oficjalna produkcja wersji 3 i 5-drzwiowej ruszyła 14 lipca 2008 roku.\cite{Wikipedia}
\pause
\begin{center}
\includegraphics[scale=0.15]{8.jpg}\cite{zdj8}
\end{center}
\end{frame}

\begin{thebibliography}{4}
\bibitem{Wikipedia}
\url{https://www.wikipedia.org/}
\bibitem{Wikipedia}
\url{https://www.wikipedia.org/}
\bibitem{Wikipedia}
\url{https://www.wikipedia.org/}
\bibitem{zdj1}
\url{https://www.wikipedia.org/}
\end{thebibliography}

\end{document}